\documentclass{article}


\usepackage{hyperref}
\hypersetup{
    colorlinks,
    citecolor=black,
    filecolor=black,
    linkcolor=black,
    urlcolor=black
}

\usepackage{graphicx}
\usepackage{tikz}
\usepackage[framemethod=TikZ]{mdframed}
\usepackage{geometry}
\usepackage{amsmath}
\usepackage{amssymb}
\usepackage{amsfonts}
\usepackage{scrextend}
\usepackage{enumitem, multicol}
\usepackage{caption}
\usepackage{ifthen}
\usepackage{changepage}
\usepackage{listings}
\usepackage{tikz-cd}
\usetikzlibrary{arrows.meta, automata, 
                positioning, 
                quotes}
\usepackage{amsmath}
\usepackage[low-sup]{subdepth}
\usepackage{ebproof}
\usepackage{mdframed}
\usepackage{calc}
\usepackage{mathtools}
\usepackage{forest}
\usepackage[dvipsnames]{xcolor}


\newlength\twound
\settowidth{\twound}{\_}
\newcommand{\dunder}{\rule{2\twound}{0.4pt}}

\newcommand*{\fullref}[1]{\hyperref[{#1}]{\ref*{#1} \nameref*{#1}}}

\DeclarePairedDelimiter{\ceil}{\lceil}{\rceil}

\DeclarePairedDelimiter{\encode}{\ulcorner}{\urcorner}


\makeatletter
\newsavebox{\@brx}
\newcommand{\llangle}[1][]{\savebox{\@brx}{\(\m@th{#1\langle}\)}%
  \mathopen{\copy\@brx\kern-0.5\wd\@brx\usebox{\@brx}}}
\newcommand{\rrangle}[1][]{\savebox{\@brx}{\(\m@th{#1\rangle}\)}%
  \mathclose{\copy\@brx\kern-0.5\wd\@brx\usebox{\@brx}}}
\makeatother

\lstset{
    basicstyle=\ttfamily,
    mathescape=true,          % Enable math mode inside listings
    columns=fullflexible      % Ensure spacing is preserved
}

\newmdenv[
  topline=false,
  bottomline=false,
  rightline=false,
  skipabove=\topsep,
  skipbelow=\topsep
]{siderules}

\def\axiom{\infer0}

\usetikzlibrary{calc}
% `pathmorphing` is necessary to draw squiggly arrows.
\usetikzlibrary{decorations.pathmorphing,shapes.misc}


% A TikZ style for curved arrows of a fixed height, due to AndréC.
\tikzset{curve/.style={settings={#1},to path={(\tikztostart)
    .. controls ($(\tikztostart)!\pv{pos}!(\tikztotarget)!\pv{height}!270:(\tikztotarget)$)
    and ($(\tikztostart)!1-\pv{pos}!(\tikztotarget)!\pv{height}!270:(\tikztotarget)$)
    .. (\tikztotarget)\tikztonodes}},
    settings/.code={\tikzset{quiver/.cd,#1}
        \def\pv##1{\pgfkeysvalueof{/tikz/quiver/##1}}},
    quiver/.cd,pos/.initial=0.35,height/.initial=0}

% TikZ arrowhead/tail styles.
\tikzset{tail reversed/.code={\pgfsetarrowsstart{tikzcd to}}}
\tikzset{2tail/.code={\pgfsetarrowsstart{Implies[reversed]}}}
\tikzset{2tail reversed/.code={\pgfsetarrowsstart{Implies}}}
% TikZ arrow styles.
\tikzset{no body/.style={/tikz/dash pattern=on 0 off 1mm}}

\definecolor{dkgreen}{rgb}{0,0.6,0}
\definecolor{darkred}{rgb}{0.55,0,0}
\definecolor{gray}{rgb}{0.5,0.5,0.5}
\definecolor{mauve}{rgb}{0.58,0,0.82}

%\usetikzlibrary{decorations.markings}

\geometry{
 a4paper,
 total={170mm,257mm},
 left=30mm,
 top=25mm,
}

\newcommand{\indrule}[3]{\ifstrempty{#1}{\mathrel{\raisebox{6pt}{\(\dfrac{#2}{#3}\)}}}{\mathrel{\raisebox{6pt}{(\text{#1})\ \ \(\dfrac{#2}{#3}\)}}}}
\newcommand{\eqdef}{\overset{\text{def}}{=}}

\definecolor{darkblue}{rgb}{0,0,0.55}

\newcounter{quest}\setcounter{quest}{0}
\renewcommand{\thequest}{\arabic{quest}}
\newenvironment{quest}[1][]{%
\ifstrempty{#1}%
{%
\refstepcounter{quest}%
\mdfsetup{%
frametitle={%
\tikz[baseline=(current bounding box.east),outer sep=0pt]
\node[anchor=east,rectangle,fill=darkblue!20]
{\strut Question~\thequest.};}}
%
}%
{
\ifthenelse{\equal{#1}{a} \OR \equal{#1}{i} \OR \equal{#1}{1}}{\refstepcounter{quest}}{}
\mdfsetup{%
frametitle={%
\tikz[baseline=(current bounding box.east),outer sep=0pt]
\node[anchor=east,rectangle,fill=darkblue!20]
{\strut Question~\thequest.#1)};}}%
}%
\mdfsetup{innertopmargin=10pt,linecolor=darkblue!20,%
linewidth=2pt,topline=true,%
frametitleaboveskip=\dimexpr-\ht\strutbox\relax
}
\begin{mdframed}[]\relax%
}{\end{mdframed}}



\setlength{\parindent}{0px}

\usetikzlibrary{decorations.markings}


\geometry{
 a4paper,
 total={170mm,257mm},
 left=30mm,
 top=25mm,
}


\newcommand{\R}{\mathbb{R}}
\newcommand{\pdv}[2]{\frac{\partial #1}{\partial #2}}
\newcommand{\lbold}[1]{\textbf{\large{ #1}}}
\newcommand{\minus}{\scalebox{1.0}[1.0]{\(-\)}}
\newcommand{\boldminus}{\(\boldsymbol{-}\)}
\newcommand{\addtopstrut}{\rule[5pt]{0pt}{.7\baselineskip}}%
\newcommand{\addbottomstrut}{\rule[-5pt]{0pt}{0pt}}%

\newcounter{theo}[section]\setcounter{theo}{0}
\renewcommand{\thetheo}{\arabic{section}.\arabic{theo}}
\newenvironment{theo}[2][]{%
\refstepcounter{theo}%
\ifstrempty{#1}%
{\mdfsetup{%
frametitle={%
\tikz[baseline=(current bounding box.east),outer sep=0pt]
\node[anchor=east,rectangle,fill=red!20]
{\strut Theorem~\thetheo};}}
}%
{\mdfsetup{%
frametitle={%
\tikz[baseline=(current bounding box.east),outer sep=0pt]
\node[anchor=east,rectangle,fill=red!20]
{\strut Theorem~\thetheo:~#1};}}%
}%
\mdfsetup{innertopmargin=10pt,linecolor=red!20,%
linewidth=2pt,topline=true,%
frametitleaboveskip=\dimexpr-\ht\strutbox\relax
}
\begin{mdframed}[]\relax%
\label{#2}}{\end{mdframed}}

\definecolor{pastelgreen}{RGB}{193, 225, 193}

\newcounter{lem}[section]\setcounter{lem}{0}
\renewcommand{\thelem}{\arabic{section}.\arabic{lem}}
\newenvironment{lem}[2][]{%
\refstepcounter{lem}%
\ifstrempty{#1}%
{\mdfsetup{%
frametitle={%
\tikz[baseline=(current bounding box.east),outer sep=0pt]
\node[anchor=east,rectangle,fill=pastelgreen]
{\strut Lemma~\thelem};}}
}%
{\mdfsetup{%
frametitle={%
\tikz[baseline=(current bounding box.east),outer sep=0pt]
\node[anchor=east,rectangle,fill=pastelgreen]
{\strut Lemma~\thelem:~#1};}}%
}%
\mdfsetup{innertopmargin=10pt,linecolor=pastelgreen,%
linewidth=2pt,topline=true,%
frametitleaboveskip=\dimexpr-\ht\strutbox\relax
}
\begin{mdframed}[]\relax%
\label{#2}}{\end{mdframed}}

\definecolor{darkblue}{rgb}{0,0,0.55}

\newcounter{defin}[section]\setcounter{defin}{0}
\renewcommand{\thedefin}{\arabic{section}.\arabic{defin}}
\newenvironment{defin}[2][]{%
\refstepcounter{defin}%
\ifstrempty{#1}%
{\mdfsetup{%
frametitle={%
\tikz[baseline=(current bounding box.east),outer sep=0pt]
\node[anchor=east,rectangle,fill=darkblue!20]
{\strut Definition~\thedefin};}}
}%
{\mdfsetup{%
frametitle={%
\tikz[baseline=(current bounding box.east),outer sep=0pt]
\node[anchor=east,rectangle,fill=darkblue!20]
{\strut Definition~\thedefin:~#1};}}%
}%
\mdfsetup{innertopmargin=10pt,linecolor=darkblue!20,%
linewidth=2pt,topline=true,%
frametitleaboveskip=\dimexpr-\ht\strutbox\relax
}
\begin{mdframed}[]\relax%
\label{#2}}{\end{mdframed}}

\newcounter{prf}[section]\setcounter{prf}{0}
\renewcommand{\theprf}{\arabic{section}.\arabic{prf}}
\newenvironment{prf}[2][]{%
\refstepcounter{prf}%
\ifstrempty{#1}%
{\mdfsetup{%
frametitle={%
\tikz[baseline=(current bounding box.east),outer sep=0pt]
\node[anchor=east,rectangle,fill=red!20]
{\strut Proof~\theprf};}}
}%
{\mdfsetup{%
frametitle={%
\tikz[baseline=(current bounding box.east),outer sep=0pt]
\node[anchor=east,rectangle,fill=red!20]
{\strut Proof~\thetheo:~#1};}}%
}%
\mdfsetup{innertopmargin=10pt,linecolor=red!20,%
linewidth=2pt,topline=true,%
frametitleaboveskip=\dimexpr-\ht\strutbox\relax
}
\begin{mdframed}[]\relax%
\label{#2}}{\end{mdframed}}



\setlength{\parindent}{0px}

\tikzset{
  ctrlpoint/.style={%
    draw=gray,
    circle,
    inner sep=0,
    minimum width=1ex,
  }
}

% cSpell:disable

\newcommand\Bezier[4]{% \bezier (lowercase 'b') was already defined elsewhere
  \node (p1) [ctrlpoint,label=90:\(P_1\)] at (#1) {};
  \node (p2) [ctrlpoint,label=90:\(P_2\)] at (#2) {};
  \node (p3) [ctrlpoint,label=90:\(P_3\)] at (#3) {};
  \node (p4) [ctrlpoint,label=90:\(P_4\)] at (#4) {};
  \draw [gray] (p1) -- (p2) -- (p3) -- (p4);
  \draw [blue] (#1) .. controls (#2) and (#3) .. (#4);
}

% cSpell:enable
\usepackage{adjustbox}



\title{Semantics of Programming Languages}
\author{Simon}
\begin{document}
\maketitle

\tableofcontents
\newpage

\section{First Imperative Language (L1)}

\subsection{L1 - Syntax}

L1 has the following syntax:
\begin{alignat*}{2}
    &\text{Booleans } \qquad &&b \in \mathbb{B}\ =\ \{\text{\textbf{true}}, \text{\textbf{false}}\}\\
    &\text{Integers } &&n \in \mathbb{Z}\\
    &\text{Locations} &&\ell \in \mathbb{L}\ =\ \{l, l_0, l_1, l_2, \dots\}\\[10px]
    &\text{Operations} &&op ::= + \,|\, \ge\\
    &\text{Expressions}
\end{alignat*}
\begin{alignat*}{1}
    e ::= &n \ |\  b \ |\  e_{1} \ op \ e_{2} \ |\  \text{\textbf{if}}\ e_{1} \text{\textbf{then}}\ e_{2}\ \text{\textbf{else}}\ e_{3} \ |\ \\
    &\ell \coloneq e \ |\  !l \ |\ \\
    &\text{\textbf{skip}} \ |\  e_{1};e_{2} \ |\ \\
    &\text{\textbf{while}}\ e_{1}\ \text{\textbf{do}}\ e_{2}
\end{alignat*}

\subsection{Operational Semantics}

In order to describe the behaviour of L1 programs we will use structural operational semantics to define various forms of automata.

\vspace{5px}

A transition system consists of 
\begin{itemize}
    \item A set \textbf{Config} 
    \item A binary relation \( \longrightarrow \ \subseteq \text{\textbf{Config}} \times \text{\textbf{Config}}\)
\end{itemize}

\textbf{Notation}:

\begin{itemize}
    \item \( \longrightarrow^*\) is the reflexive transitive closure of \( \longrightarrow \), so \(c \longrightarrow c'\) iff there exists \(k \geq 0\) and
    \(c_0, \dots, c_k\) such that \(c = c_0 \longrightarrow c_{1} \longrightarrow \dots \longrightarrow  c_k = c'\) 
    \item \(\not \longrightarrow \) is a unary predicate (a subset of \textbf{Config}) defined by \(c \not \longrightarrow \) iff \(\neg \exists\ c'. c \longrightarrow c\)
\end{itemize}

We say there are finite stores \(s \in \mathbb{L} \rightharpoonup \mathbb{Z}\). And then define configurations to be pairs \(\langle e, s \rangle\) of an expression \(e\),
and a store \(s\), which means our transition relation will have the form:

\[
    \langle e, s \rangle \longrightarrow \langle e', s' \rangle 
\]

We can continue transitioning until we reach a value:

\[
    \mathbb{V} = \mathbb{B}\ \cup\ \mathbb{Z}\ \cup\ \{\text{\textbf{skip}}\}
\]
\[
    v ::= b \ | \ n \ | \ \text{\textbf{skip}} \quad \ \ \ \ \ 
\]

\vspace{5px}

\textbf{Stuck}

Call a configuration \(c = \langle e, s \rangle \) stuck if \(\ \ e \not \in \mathbb{V}\ \land\ \langle e, s \rangle \not \longrightarrow \)

\newpage

\subsection{L1 Full Operational Semantics}\label{L1-collected-semantics}

\[
\begin{aligned}
& (\mathrm{op}\ +) \quad\left\langle n_1+n_2, s\right\rangle \longrightarrow\langle n, s\rangle \quad \qquad \text { if } n=n_1+n_2 \\[5px]
\ \ \ \ & (\mathrm{op} \geq) \quad\left\langle n_1 \geq n_2, s\right\rangle \longrightarrow\langle b, s\rangle \qquad \ \  \ \, \text { if } b=\left(n_1 \geq n_2\right) \\[5px]
& (\mathrm{op} 1)\ \ \ \frac{\left\langle e_1, s\right\rangle \longrightarrow\left\langle e_1^{\prime}, s^{\prime}\right\rangle}{\left\langle e_1 \text { op } e_2, s\right\rangle \longrightarrow\left\langle e_1^{\prime} \text { op } e_2, s^{\prime}\right\rangle} \\[10px]
& (\mathrm{op2})\ \ \ \frac{\left\langle e_2, s\right\rangle \longrightarrow\left\langle e_2^{\prime}, s^{\prime}\right\rangle}{\left\langle v \text { op } e_2, s\right\rangle \longrightarrow\left\langle v \quad o p \quad e_2^{\prime}, s^{\prime}\right\rangle}
\end{aligned}
\]


\begin{alignat*}{3}
&\text { (deref) }\qquad &&\langle!\ell, s\rangle \longrightarrow\langle n, s\rangle \quad &&\text { if } \ell \in \operatorname{dom}(s) \text { and } s(\ell)=n\\[5px]
&\text { (assign1) }&&\langle\ell:=n, s\rangle \longrightarrow\langle\text { \textbf{skip}, } s+\{\ell \mapsto n\}\rangle \quad &&\text { if } \ell \in \operatorname{dom}(s)\\[10px]
&\text { (assign2) } &&\frac{\langle e, s\rangle \longrightarrow\left\langle e^{\prime}, s^{\prime}\right\rangle}{\langle\ell:=e, s\rangle \longrightarrow\left\langle\ell:=e^{\prime}, s^{\prime}\right\rangle}
\end{alignat*}


\[
\text { (seq1) }\left\langle\text { skip } ; e_2, s\right\rangle \longrightarrow\left\langle e_2, s\right\rangle \phantom{aaaaaaaaaaaaaaaaa}
\]

\[
\text { (seq2) } \frac{\left\langle e_1, s\right\rangle \longrightarrow\left\langle e_1^{\prime}, s^{\prime}\right\rangle}{\left\langle e_1 ; e_2, s\right\rangle \longrightarrow\left\langle e_1^{\prime} ; e_2, s^{\prime}\right\rangle} \phantom{aaaaaaaaaaaaaaaaa}
\]

\begin{alignat*}{2}
    &( \text{if}1 ) \quad \langle \text{\textbf{if}}\ \text{\textbf{true}}\  &&\text{\textbf{then}}\ e_{2}\ \text{\textbf{else}}\ e_{3},\ s \rangle \longrightarrow \langle e_{2},\ s \rangle\\[5px]
    &( \text{if}1 ) \quad \langle \text{\textbf{if}}\ \text{\textbf{false}}\ &&\text{\textbf{then}}\ e_{2}\ \text{\textbf{else}}\ e_{3},\ s \rangle \longrightarrow \langle e_{2},\ s \rangle
\end{alignat*}

\[
\text { (if3) }\quad \frac{\left\langle e_1, s\right\rangle \longrightarrow\left\langle e_1^{\prime}, s^{\prime}\right\rangle}{\left\langle\text { if } e_1 \text { then } e_2 \text { else } e_3, s\right\rangle \longrightarrow\left\langle\text { if } e_1^{\prime} \text { then } e_2 \text { else } e_3, s^{\prime}\right\rangle} \\
\]

\[
\text{(while)}\quad\left\langle \text{\textbf{while}}\ e_1\ \text{\textbf{do}}\ e_2, s\right\rangle \longrightarrow\left\langle \text{\textbf{if}}\ e_1\ \text{\textbf{then}}\left(e_2 ;\ \text{\textbf{while}}\ e_1\ \text{\textbf{do}}\ e_2\right)\ \text{\textbf{else}}\ \text{\textbf{skip}}, s\right\rangle
\]

\newpage

\subsection{L1 Determinancy}

\begin{theo}[Determinancy]{}
    If \(\langle e,s \rangle \longrightarrow \langle e_{1}, s_{1} \rangle\) and \(\langle e,s \rangle \longrightarrow \langle e_{2}, s_{2} \rangle\) then
    \(\langle e_1, s_1 \rangle = \langle e_2, s_2 \rangle \)
\end{theo}

\begin{lem}[Irreducibility of Values]{\thelem}
    For all \(e \in L_1\), if \(e\) is a value then
    \[
        \forall s. \neg \exists e', s' \ . \ \langle e,s \rangle \longrightarrow \langle e', s; \rangle
    \]

    \textbf{Proof}:

    \vspace{5px}

    By definition of the set of values \(e\) must be in the form of one of the following \(n, b, \text{\textbf{skip}}\), by considering
    the rules of the language there is no rule with the conclusion of the form \(\langle e,s \rangle \longrightarrow  \langle e',s' \rangle\) for any
    \(e\) of the form previously described.
\end{lem}


\subsubsection{Inversion}

For inductive proofs we often need an inversion property, that given a tuple in one inductively defined relation, gives you a case
analysis of the possible last rule used.

\begin{lem}[Inversion for \( \longrightarrow \)]{\thelem}
    (take any occurences of (\(n_i, n\)) to be integers (\(\mathbb{Z}\)), (\(\ell, l_i\)) to be labels (\(\mathbb{L}\)) and (\(b, b_i\)) to be booleans \((\mathbb{B})\))

    \vspace{5px}

    If \(\langle e,s \rangle \longrightarrow \langle \hat{e}, \hat{s} \rangle\) then either:
    \vspace{10px}

    \vspace{10px}

    \begin{tabular}{c l l}
        (op \(+\)) & \(\exists n_1, n_2, n\) &s.t. \(e = n_1 + n_2\ \) and \(\ \hat{e} = n\) and \(\ n = n_{1} +\,\! n_{2}\ \) and \(\ \hat{s} = s\)\\[5px]
        (op \( \geq\)) & \(\exists n_1, n_2, b\) &s.t. \(e = n_1 \geq n_2\) and \(\hat{e} = b\) and \(\hat{s} = s\) and \(\,b = n_{1} \geq n_{2}\)\\[5px]
        (op1) & \(\exists e_{1}, e_{2}, op, e_1'\) &s.t. \(e = e_{1}\ op\ e_{2}\) and \(e = e_{1}'\ op\ e_{2}\) and \(\langle e_{1}, s \rangle \longrightarrow \langle e_{1}', s' \rangle  \)\\[5px]
        (op2) & \(\exists n, e_{2}, op, e_2'\) &s.t. \(e = n\ \,\, op\ e_{2}\) and \(e = n\ \, \, op\ e_{2}'\) and \(\langle e_{2}, s \rangle \longrightarrow \langle e_{2}', s' \rangle  \)\\[5px]
        (deref) & \(\exists \ell, n\) &s.t. \(e =\ !\ell\), \(\hat{e} = n\) and \(\hat{s} = s\) and \(\ell \in \text{dom}(s)\) and \(s(\ell ) =n\)\\[5px]
        (assign1) & \(\exists \ell, n\) &s.t. \(e = (\ell \coloneq n)\) and \(\hat{e} = \text{\textbf{skip}}\) and \(\hat{s} = s + \{\ell \mapsto n\}\)\\[5px]
        (assign2) & \(\exists \ell, e_1, e_1'\) &s.t. \(e = (\ell \coloneq e_1)\) and \(\hat{e} = (\ell \coloneq e_1')\) and \(\ell \in \text{dom}(s)\) and \(\langle e_{1}, s \rangle \longrightarrow \langle e_1', s' \rangle \)\\[5px]
        (if1) & \(\exists e_{2}, e_{3}\) &s.t. \(e = \text{\textbf{if}}\ \text{\textbf{true}}\,\ \text{\textbf{then}}\ e_{2}\ \text{\textbf{else}}\ e_{3}\)
        and \(\hat{e} = e_{2}\) and \(\hat{s} = s\)\\[5px]
        (if2) & \(\exists e_{2}, e_{3}\) &s.t. \(e = \text{\textbf{if}}\ \text{\textbf{false}}\ \text{\textbf{then}}\ e_{2}\ \text{\textbf{else}}\ e_{3}\)
        and \(\hat{e} = e_{3}\) and \(\hat{s} = s\)\\[5px]
        (if3) & \(\exists e_{1}, e_{2}, e_{3}, e_1'\) &s.t. \(e = \text{\textbf{if}}\ e_{1}\ \text{\textbf{then}}\ e_{2}\ \text{\textbf{else}}\ e_{3}\)
        and \(\hat{e} = \text{\textbf{if}}\ e_{1}'\ \text{\textbf{then}}\ e_{2}\ \text{\textbf{else}}\ e_{3}\)\\
        & & \(\quad \ \ \)and \(\langle e_{1}, s \rangle \longrightarrow \langle e_{2}, \hat{s} \rangle\)\\[5px]
        (while) & \(\exists e_{1}, e_{2}\) &s.t. \(e = \text{\textbf{while}}\ e_{1}\ \text{\textbf{do}}\ e_{2}\) and \(\hat{e} = \text{\textbf{if}}\ e_{1}\ \text{\textbf{then}}\ (e_{2};\text{\textbf{while}}\ e_{1}\ \text{\textbf{do}}\ e_{2})\ \text{\textbf{else}}\ \text{\textbf{skip}}\)\\
        & & \(\quad \ \ \)and \(\hat{s} = s \)
    \end{tabular}

    \vspace{20px}

    \textbf{Proof}:

    Let \[
        \Phi(e,s,\hat{e},\hat{s}) = \langle e,s \rangle \longrightarrow \langle \hat{e}, \hat{s} \rangle \implies ( \text{op} +)\ \lor\ ( \text{op} \geq)\ \lor\ \dots\ \lor\ ( \text{if}3)\ \lor\ ( \text{while}) 
    \]

    Assume the LHS of the implication that is there is a transition from \(\langle e,s \rangle\) to \(\langle \hat{e}, \hat{s} \rangle\) then we
    are required to prove that the transition is one of the transitions defined by the operational semantics of the L1 language.

    \vspace{5px}

    Since we are assuming there is a transition we only need to consider rules that provide a transition and that \(\langle e,s \rangle\) is of the form of
    the LHS of a transition rule.

    \newpage

    \textbf{Case 1}: \(\quad (e\) is of the form \(n_{1} + n_{2}\))

    \begin{adjustwidth}{20px}{0px}
        Take \(s\) to be arbitrary then since \(n_1, n_2 \in \mathbb{Z}\) there is a unique \(n\) s.t. \(n_1 + n_2 = n\) so we we can use the (op \(+\)) rule
        to get \(\langle n_{1}+n_{2},s \rangle \longrightarrow \langle n, s \rangle \)
    \end{adjustwidth}

    \vspace{5px}

    \textbf{Case 2}: \(\quad (e\) is of the form \(n_{1} \geq n_{2}\))

    \begin{adjustwidth}{20px}{0px}
        Take \(s\) to be arbitrary then since \(n_1, n_2 \in \mathbb{Z}\) there is a \(b\) s.t. \(n_1 \geq n_2 = b\) so we we can use the (op \(\geq\)) rule
        to get \(\langle n_{1} \geq n_{2},s \rangle \longrightarrow \langle b, s \rangle \)
    \end{adjustwidth}

    \vspace{5px}

    \textbf{Case 3}: \(\quad (e\) is of the form \(n_{1} \geq n_{2}\))

    \begin{adjustwidth}{20px}{0px}
        Take \(s\) to be arbitrary then since \(n_1, n_2 \in \mathbb{Z}\) there is a \(b\) s.t. \(n_1 \geq n_2 = b\) so we we can use the (op \(\geq\)) rule
        to get \(\langle n_{1} \geq n_{2},s \rangle \longrightarrow \langle b, s \rangle \)
    \end{adjustwidth}
\end{lem}

\newpage

\textbf{Proof} of Determinancy

\vspace{10px}


Take \[
    \Phi(e) \triangleq \forall  s, e', e'', s', s''.\ \big(\langle e,s \rangle \longrightarrow \langle e', s' \rangle \land \langle e,s \rangle \longrightarrow \langle e'', s'' \rangle  \big) \implies \langle e',s' \rangle = \langle e'', s'' \rangle 
\]

We are RTP that \(\forall e \in L_1.\ \Phi(e)\).

\vspace{10px}

\textbf{Case} \(e \in \{\text{\textbf{skip}}, b, n\}\)

\vspace{5px}

\begin{adjustwidth}{20px}{0px}
    If \(e\) is a value there are no transition rules that have a conclusion of the from \(\langle v, s \rangle \longrightarrow \langle ..,\ .. \rangle\)
    which means the LHS is false for this form meaning \(\Phi(e)\) when \(e\) is a value holds vacuously. 
\end{adjustwidth}

\vspace{10px}

For the remaining cases of the form of \(e\) we will use structural induction, and proceed by assuming the LHS of the implication and use
further case analysis to look at the type of the transition.

\vspace{10px}

\textbf{Case} \(e =\ !\ell \)

\vspace{5px}

\begin{adjustwidth}{20px}{0px}
    Take arbitrary \(s, e', e'', s', s''\) s.t. \(\langle !\ell ,s \rangle \longrightarrow \langle e', s' \rangle \land \langle !\ell ,s \rangle \longrightarrow \langle e'', s'' \rangle\),
    the only transition rule that has a conclusion with the lhs expression being of the form \(!\ell \) is (deref), so both transitions instances
    of this rule which then means.
    \vspace{-10px}
    \begin{alignat*}{2}
        &\ell \in \text{dom}(s) \qquad &&\ell \in \text{dom}(s)\\
        &e' = s(\ell) &&e'' = s(\ell)\\
        &s' = s && s'' = s
    \end{alignat*}
    So from this we have \(s' = s = s''\), and since \(s\) is a store which is a partial function mapping from the set of labels to the set of integers
    then if \(\ell \in \text{dom}(s)\) then \(e' = s(\ell) = e''\), which means \(\langle e',s' \rangle = \langle e'', s'' \rangle\), so \(\Phi\) holds
    in this case of the form of \(e\).
\end{adjustwidth}

\vspace{10px}

\textbf{Case} \(e =\ (\ell \coloneq e_{1})\quad\quad\) suppose that \(\Phi(e_{1})\) then we are RTP \(\Phi(\ell \coloneq e_{1})\)

\vspace{5px}

\begin{adjustwidth}{20px}{0px}
    Take arbitrary \(s, e', e'', s', s''\) s.t. \(\langle \ell \coloneq e_{1} ,s \rangle \longrightarrow \langle e', s' \rangle \land \langle \ell \coloneq e_{1} ,s \rangle \longrightarrow \langle e'', s'' \rangle\),
    then there are two forms for the transition they are the rules (assign1) and (assign2) wlog there are 3 cases we have for the possible permutations
    of the pairings:

    \vspace{10px}

    \textbf{Case} \(\langle \ell \coloneq e_{1} ,s \rangle \longrightarrow \langle e', s' \rangle\) is (assign1) and \(\langle \ell \coloneq e_{1} ,s \rangle \longrightarrow \langle e'', s'' \rangle\) is (assign1)

    \vspace{5px}

    \begin{adjustwidth}{20px}{0px}
        Then for both to be instances of (assign1) then \(e_{1}\) must be a value, and more specifically an integer value \(n\). That is
        \(e = (\ell \coloneq n)\), then looking at (assign1):
        \begin{alignat*}{2}
            &\ell \in \text{dom}(s) &&\ell \in \text{dom}(s)\\
            &e' = \text{\textbf{skip}} &&e'' = \text{\textbf{skip}}\\
            &s' = s + \{\ell \mapsto n\}\qquad &&s'' = s + \{\ell \mapsto n\}
        \end{alignat*}
        \[
            \implies \langle e',s' \rangle = \langle e'', s'' \rangle \qquad \
        \]
    \end{adjustwidth}

    \vspace{10px}

    \textbf{Case} \(\langle \ell \coloneq e_{1} ,s \rangle \longrightarrow \langle e', s' \rangle\) is (assign2) and \(\langle \ell \coloneq e_{1} ,s \rangle \longrightarrow \langle e'', s'' \rangle\) is (assign2)

    \vspace{5px}

    \begin{adjustwidth}{20px}{0px}
        Then \(\langle \ell \coloneq e_{1} ,s \rangle \longrightarrow \langle \ell \coloneq e_1', s' \rangle\) with \(\langle e_{1}, s \rangle \longrightarrow \langle e_{1}, s' \rangle \)
        and \(\langle \ell \coloneq e_{1} ,s \rangle \longrightarrow \langle \ell \coloneq e_1'', s' \rangle\) with \(\langle e_{1}, s \rangle \longrightarrow \langle e_{1}'', s'' \rangle \).
        By the inductive hypothesis, \(\Phi(e_{1})\) so we have that \(\langle e_{1}', s' \rangle = \langle e_{1}'', s'' \rangle\), which means that
        \begin{alignat*}{1}
            &\implies \langle \ell \coloneq e_1', s' \rangle = \langle \ell \coloneq e_1'', s' \rangle\\
            &\implies \langle e',s' \rangle = \langle e'', s'' \rangle \qquad
        \end{alignat*}
    \end{adjustwidth}

    \vspace{10px}

    \textbf{Case} \(\langle \ell \coloneq e_{1} ,s \rangle \longrightarrow \langle e', s' \rangle\) is (assign2) and \(\langle \ell \coloneq e_{1} ,s \rangle \longrightarrow \langle e'', s'' \rangle\) is (assign1)

    \vspace{5px}

    \begin{adjustwidth}{20px}{0px}
        Then from the first transition which is an instance of (assign2) we get \(\exists e_1', s'. \ \ \langle e_{1}, s \rangle \longrightarrow \langle e_{1}', s' \rangle\), however
        we get from the second transition which is an instance of (assign1) that \(e_1\) is value and from the Irreducibility of Values lemma we have
        \(\not\exists e_1', s'. \ \ \langle e_{1}, s \rangle \longrightarrow \langle e_{1}', s' \rangle\), which is a contradiction so the assumption is false
        which means the rhs holds vacuously.
    \end{adjustwidth}

\end{adjustwidth}

\newpage


\textbf{Case} \(e = (e_{1}\ op\ e_{2})\quad\quad\) suppose that \(\Phi(e_1) \land \Phi(e_2)\) then we are RTP \(\Phi(e_{1}\ op\ e_{2})\)

\begin{adjustwidth}{20px}{0px}
    Take arbitrary \(s, e', e'', s', s''\) s.t. \(\langle e_{1}\ op\ e_{2} ,s \rangle \longrightarrow \langle e', s' \rangle \land \langle e_{1}\ op\ e_{2} ,s \rangle \longrightarrow \langle e'', s'' \rangle\),

\end{adjustwidth}


\newpage

\subsection{L1 Type System}

Define a ternary relation \(\vdash\), that with infix notation reads \(\Gamma \vdash e: T\) as \(e\) has type \(T\), under the assumptions \(\Gamma\)
on the types of locations that may occur in \(e\).

\vspace{10px}

We write \(T\) and \(T_{loc}\) for the sets of all terms of these grammars.

\vspace{5px}

And let \(\Gamma\) range over \(\text{\textbf{TypeEnv}}\), the finite partial functions from locations \(\mathbb{L}\) to \(T_{loc}\). 

(Notation: write \(\Gamma\) as \(\ \ l_1: \text{intref}, \dots, l_k: \text{intref}\ \ \) instead of \(\ \ \{l_1 \mapsto \text{intref}, \dots, l_k \mapsto \text{intref}\}\ \ \)) 

\subsection{L1 Collected Typing System}\label{L1-collected-types}

Types of expressions:
\[
    T \quad ::= \quad \text{int} \ | \ \text{bool} \ | \ \text{unit}
\]

Types of locations:
\[
    T_{loc}\quad ::=\quad \text{intref}
\]

\begin{alignat*}{3}
    &\ ( \text{int} )\quad &&\Gamma \vdash n: \text{int}\qquad &&\text{for } n \in \mathbb{Z}\\
    &( \text{bool} )\quad &&\Gamma \vdash b\,: \text{bool} &&\text{for } b \,\in \mathbb{B}
\end{alignat*}

\[
    \qquad\qquad
    \begin{prooftree}
        \hypo{\Gamma \vdash e_{1}: \text{int}}
        \hypo{\Gamma \vdash e_{2}: \text{int}}
        \infer[left label=(op+)]2{\Gamma \vdash e_{1} + e_{2}: \text{int}}
    \end{prooftree}\qquad\qquad
    \begin{prooftree}
        \hypo{\Gamma \vdash e_{1}: \text{int}}
        \hypo{\Gamma \vdash e_{2}: \text{int}}
        \infer[left label=(op\( \geq\))]2{\Gamma \vdash e_{1} \geq e_{2}: \text{bool}}
    \end{prooftree}
\]

\[
    \begin{prooftree}
        \hypo{\Gamma \vdash e_1: \text{bool}}
        \hypo{\Gamma \vdash e_{2}: T}
        \hypo{\Gamma \vdash e_{3}: T}
        \infer[left label=(if)]3{\Gamma \vdash \text{\textbf{if}}\ e_{1}\ \text{\textbf{then}}\ e_{2}\ \text{\textbf{else}}\ e_{3}: T}
    \end{prooftree}
\]

\[
    \begin{prooftree}
        \hypo{\Gamma(\ell) = \text{intref}}
        \hypo{\Gamma \vdash e: \text{int}}
        \infer[left label=(assign)]2{\Gamma \vdash \ell \coloneq e: \text{unit}}
    \end{prooftree}
\]

\[
    \begin{prooftree}
        \hypo{\Gamma(\ell) = \text{intref}}
        \infer[left label=(deref)]1{\Gamma \vdash !\ell: \text{int}}
    \end{prooftree}
\]

\[
    ( \text{skip} )\quad \Gamma \vdash \text{\textbf{skip}}: \text{unit}
\]

\[
    \begin{prooftree}
        \hypo{\Gamma \vdash e_{1}: \text{unit}}
        \hypo{\Gamma \vdash e_{2}: T}
        \infer[left label=(seq)]2{\Gamma \vdash e_{1};e_{2}: T}
    \end{prooftree}
\]

\[
    \begin{prooftree}
        \hypo{\Gamma \vdash e_{1}: \text{bool}}
        \hypo{\Gamma \vdash e_{2}: \text{unit}}
        \infer[left label=(while)]2{\Gamma \vdash \text{\textbf{while}}\ e_{1}\ \text{\textbf{do}}\ e_{2}: \text{unit}}
    \end{prooftree}
\]

\newpage

\section{Functions - L2}

\subsection{Concrete Syntax}

By convention, application associates to the left, so \(\ \ e_{1}\ e_{2}\ e_{3}\ \ \) denotes \(\ \ (e_{1}\ e_{2})\ e_{3}\ \ \) whereas type
arrows associate to the right so \(\ \ T_{1} \to T_{2} \to T_{3}\ \ \) denotes \(\ \ T_{1} \to (T_{2} \to T_{3})\ \ \).

\vspace{5px}

A \(\text{\textbf{fn}}\) extends to the right as far as parantheses permit, so \(\ \ \text{\textbf{fn}}\ x: \text{unit} \Rightarrow x;x\ \ \) denotes
\(\ \ \text{\textbf{fn}}\ x: \text{unit} \Rightarrow (x;x)\ \ \) 

\begin{itemize}
    \item Variables are not locations \((\mathbb{L} \cap \mathbb{X} = \{\})\) 
    \item Cannot abstract on locations. For example, (\(\text{\textbf{fn}}\ \ell: \text{intref} \Rightarrow !\ell \)) is not in the syntax
\end{itemize}


\subsection{Alpha Conversion}

In expressions \( \text{\textbf{fn}}\ x: T \Rightarrow e\) the \(x\) is a binder

\begin{itemize}
    \item Inside \(e\), any \(x\)'s (that themselves are not binder and are not inside another \(\text{\textbf{fn}}\ x: T' \dots\)) mean the
    same thing - the formal paramater of this function. 
    \item Outside this \( \text{\textbf{fn}}\ x: T \Rightarrow e\) it does not matter which variable we used for the formal paramater   
\end{itemize}

We will allow ourselves to at any time at all, in any expression replace the binding \(x\) and all occurences of \(x\) that are bound by that 
binder, by any other variable - so long as that does not change the binding graph.

\[
    \text{\textbf{fn}}\ z : \text{int} \to \text{int} \to \text{int} \Rightarrow (\text{\textbf{fn}}\ y: \text{int} \Rightarrow z\ y\ y)
\]

% https://q.uiver.app/#q=WzAsNyxbMiwxLCJcXHRleHR7XFx0ZXh0YmZ7Zm4gfX0gXFxidWxsZXQ6IFxcdGV4dHtpbnR9IFxcUmlnaHRhcnJvdyJdLFsyLDIsIkAiXSxbMSwzLCJAIl0sWzAsNCwiXFxidWxsZXQiXSxbMiw0LCJcXGJ1bGxldCJdLFszLDMsIlxcYnVsbGV0Il0sWzIsMCwiXFx0ZXh0e1xcdGV4dGJme2ZuIH19IFxcYnVsbGV0OiBcXHRleHR7aW50fSBcXHRvIFxcdGV4dHtpbnR9IFxcdG8gXFx0ZXh0e2ludCB9IFxcUmlnaHRhcnJvdyJdLFswLDEsIiIsMCx7InN0eWxlIjp7ImhlYWQiOnsibmFtZSI6Im5vbmUifX19XSxbMSwyLCIiLDAseyJzdHlsZSI6eyJoZWFkIjp7Im5hbWUiOiJub25lIn19fV0sWzIsMywiIiwwLHsic3R5bGUiOnsiaGVhZCI6eyJuYW1lIjoibm9uZSJ9fX1dLFsyLDQsIiIsMCx7InN0eWxlIjp7ImhlYWQiOnsibmFtZSI6Im5vbmUifX19XSxbMSw1LCIiLDAseyJzdHlsZSI6eyJoZWFkIjp7Im5hbWUiOiJub25lIn19fV0sWzUsMCwiIiwwLHsiY3VydmUiOjQsImNvbG91ciI6WzAsNjAsNjBdfV0sWzQsMCwiIiwwLHsiY3VydmUiOjQsImNvbG91ciI6WzAsNjAsNjBdfV0sWzMsNiwiIiwwLHsiY3VydmUiOi00LCJjb2xvdXIiOlswLDYwLDYwXX1dLFs2LDAsIiIsMCx7InN0eWxlIjp7ImhlYWQiOnsibmFtZSI6Im5vbmUifX19XV0=
\[\begin{tikzcd}
	&& {\text{\textbf{fn }} \cdot: \text{int} \to \text{int} \to \text{int } \Rightarrow} \\
	&& {\text{\textbf{fn }} \cdot: \text{int} \Rightarrow} \\
	&& {@} \\
	& {@} && \bullet \\
	\bullet && \bullet
	\arrow[no head, from=1-3, to=2-3]
	\arrow[no head, from=2-3, to=3-3]
	\arrow[no head, from=3-3, to=4-2]
	\arrow[no head, from=3-3, to=4-4]
	\arrow[no head, from=4-2, to=5-1]
	\arrow[no head, from=4-2, to=5-3]
	\arrow[color={rgb,255:red,214;green,92;blue,92}, curve={height=24pt}, from=4-4, to=2-3]
	\arrow[color={rgb,255:red,214;green,92;blue,92}, curve={height=-24pt}, from=5-1, to=1-3]
	\arrow[color={rgb,255:red,214;green,92;blue,92}, curve={height=24pt}, from=5-3, to=2-3]
\end{tikzcd}\]

\subsubsection{De Brujin Indices}

Our implementation will use those pointers - known as De Brujin indices. Each occurence of a bound variable is represented by the number of \(\text{\textbf{fn }} \cdot: T \Rightarrow\) nodes you
have to count out to get to its binder.

\vspace{10px}

\begin{center}
\begin{minipage}{0.435\textwidth}
    \[
        \text{\textbf{fn }} \cdot: \text{int} \Rightarrow (\text{\textbf{fn}}\ \cdot: \text{int} \Rightarrow v_0 + 2)
    \]

    \vspace{10px}

    % https://q.uiver.app/#q=WzAsNSxbMSwwLCJcXHRleHR7XFx0ZXh0YmZ7Zm4gfX0gXFxidWxsZXQ6IFxcdGV4dHtpbnR9IFxcUmlnaHRhcnJvdyJdLFsxLDEsIlxcdGV4dHtcXHRleHRiZntmbiB9fSBcXGJ1bGxldDogXFx0ZXh0e2ludH0gXFxSaWdodGFycm93Il0sWzEsMiwiKyJdLFswLDMsIlxcYnVsbGV0Il0sWzIsMywiMiJdLFswLDEsIiIsMCx7InN0eWxlIjp7ImhlYWQiOnsibmFtZSI6Im5vbmUifX19XSxbMSwyLCIiLDAseyJzdHlsZSI6eyJoZWFkIjp7Im5hbWUiOiJub25lIn19fV0sWzIsMywiIiwwLHsic3R5bGUiOnsiaGVhZCI6eyJuYW1lIjoibm9uZSJ9fX1dLFsyLDQsIiIsMCx7InN0eWxlIjp7ImhlYWQiOnsibmFtZSI6Im5vbmUifX19XSxbMywxLCIiLDAseyJjdXJ2ZSI6LTJ9XV0=
\[\begin{tikzcd}
	& {\text{\textbf{fn }} \bullet: \text{int} \Rightarrow} \\
	& {\text{\textbf{fn }} \bullet: \text{int} \Rightarrow} \\
	& {+} \\
	\bullet && 2
	\arrow[no head, from=1-2, to=2-2]
	\arrow[no head, from=2-2, to=3-2]
	\arrow[no head, from=3-2, to=4-1]
	\arrow[no head, from=3-2, to=4-3]
	\arrow[color={rgb,255:red,214;green,92;blue,92}, curve={height=-12pt}, from=4-1, to=2-2]
\end{tikzcd}\]
\end{minipage}
\(\ne\)
\begin{minipage}{0.435\textwidth}
    \[
        \text{\textbf{fn }} \cdot: \text{int} \Rightarrow (\text{\textbf{fn}}\ \cdot: \text{int} \Rightarrow v_1 + 2)
    \]

    \vspace{10px}

    % https://q.uiver.app/#q=WzAsNSxbMSwwLCJcXHRleHR7XFx0ZXh0YmZ7Zm4gfX0gXFxidWxsZXQ6IFxcdGV4dHtpbnR9IFxcUmlnaHRhcnJvdyJdLFsxLDEsIlxcdGV4dHtcXHRleHRiZntmbiB9fSBcXGJ1bGxldDogXFx0ZXh0e2ludH0gXFxSaWdodGFycm93Il0sWzEsMiwiKyJdLFswLDMsIlxcYnVsbGV0Il0sWzIsMywiMiJdLFswLDEsIiIsMCx7InN0eWxlIjp7ImhlYWQiOnsibmFtZSI6Im5vbmUifX19XSxbMSwyLCIiLDAseyJzdHlsZSI6eyJoZWFkIjp7Im5hbWUiOiJub25lIn19fV0sWzIsMywiIiwwLHsic3R5bGUiOnsiaGVhZCI6eyJuYW1lIjoibm9uZSJ9fX1dLFsyLDQsIiIsMCx7InN0eWxlIjp7ImhlYWQiOnsibmFtZSI6Im5vbmUifX19XSxbMywxLCIiLDAseyJjdXJ2ZSI6LTJ9XV0=
\[\begin{tikzcd}
	& {\text{\textbf{fn }} \bullet: \text{int} \Rightarrow} \\
	& {\text{\textbf{fn }} \bullet: \text{int} \Rightarrow} \\
	& {+} \\
	\bullet && 2
	\arrow[no head, from=1-2, to=2-2]
	\arrow[no head, from=2-2, to=3-2]
	\arrow[no head, from=3-2, to=4-1]
	\arrow[no head, from=3-2, to=4-3]
	\arrow[color={rgb,255:red,214;green,92;blue,92}, curve={height=-24pt}, from=4-1, to=1-2]
\end{tikzcd}\]
\end{minipage}
\end{center}

\newpage

\subsection{Free Variables}

Say the free variables of an expression \(e\) are the set of variables \(x\) for which there is an occurence of \(x\) free in \(e\)

\begin{alignat*}{2}
    & \text{\textbf{fv}}(n) &&=\ \ \{\}\\
    & \text{\textbf{fv}}(b) &&=\ \ \{\}\\
    & \text{\textbf{fv}}(!\ell ) &&=\ \ \{\}\\
    & \text{\textbf{fv}}(\text{\textbf{skip}}) &&=\ \ \{\}\\
    & \text{\textbf{fv}}(x) &&=\ \ \{x\}\\
    & \text{\textbf{fv}}(\ell \coloneq e) &&=\ \ \text{\textbf{fv}}(e)\\
    & \text{\textbf{fv}}( \text{\textbf{fn}}\ x: T \Rightarrow e) \ \ &&=\ \ \text{\textbf{fv}}(e) - \{x\}\\
    & \text{\textbf{fv}}(e_{1}\ \ e_{2}) &&=\ \ \text{\textbf{fv}}(e_{1}) \cup \text{\textbf{fv}}(e_{2})\\
    & \text{\textbf{fv}}(e_{1}\ op\ e_{2}) &&=\ \ \text{\textbf{fv}}(e_{1}) \cup \text{\textbf{fv}}(e_{2})\\
    & \text{\textbf{fv}}(e_{1}; e_{2}) &&=\ \ \text{\textbf{fv}}(e_{1}) \cup \text{\textbf{fv}}(e_{2})\\
    & \text{\textbf{fv}}(\text{\textbf{while}}\ e_{1}\ \text{\textbf{do}}\ e_{2}) &&=\ \ \text{\textbf{fv}}(e_{1}) \cup \text{\textbf{fv}}(e_{2})\\
    & \text{\textbf{fv}}(\text{\textbf{if}}\ e_{1}\ \text{\textbf{then}}\ e_{2}\ \text{\textbf{else}}\ e_{3})\ \ &&=\ \ \text{\textbf{fv}}(e_{1}) \cup \text{\textbf{fv}}(e_{2}) \cup \text{\textbf{fv}}(e_{3})\\
\end{alignat*}

We say \(e\) is closed if \(\ \ \text{\textbf{fv}}(e) = \{\}\)

\vspace{5px}

If \(E\) is a set of expressions, write \(\text{\textbf{fv}}(E)\) for \(\displaystyle \bigcup_{e\ \in\ E} \text{\textbf{fv}}(e)\)


\subsubsection{Substitution}

The semantics for functions will involve substituting actual paramaters for formal parameters.

\vspace{5px}

Write \(\{e / x\}e'\) (Can also use the notation from Computation Theory, \(e'[e / x]\)) for the result of substituting \(e\) for all free occurences of \(x\) in \(e'\)

\begin{alignat*}{3}
    &\{e / z\}x &&=\ \ \begin{cases}
        \ e & \text{if}\ x = z\\
        \ x\quad & \text{otherwise}
    \end{cases}\\
    &\{e / z\} \big(\text{\textbf{fn}}\ x: T \Rightarrow e_1\big)\ \ &&=\ \ \text{\textbf{fn}}\ x: T \Rightarrow \big(\{e / z\} e_{1}\big) \qquad &&\text{if}\ \ \ \ x \neq z\ \ \ \ \ \ (^*)\\
    & && && \text{and}\ x \not\in \text{\textbf{fv}}(e) \ (^*)\\
    &\{e / z\}(e_{1}\ e_{2}) &&= \ \ \big(\{e / z\}e_{1}\big)\ \big(\{e / z\}e_{2}\big)\\
    &\ \ \ \ \ \ \ \vdots &&\phantom{aaaaaaaaaaaa} \vdots
\end{alignat*}

If \((^*)\) is not true, we first have to pick an alpha-variant of \(\text{\textbf{fn}}\ x: T \Rightarrow e_{1}\) to make it true

\subsubsection{Simultaneous Substitution}

A substitution \(\sigma\) is a finite partial function from variables to expressions.

\vspace{5px}

Write \(\sigma\) as \(\{e_1 / x_1, \dots, e_k / x_k\}\) instead of \(\{x_1 \mapsto e_{1}, \dots x_k \mapsto e_{k}\}\)

\vspace{10px}

Write \( \text{dom}(\sigma)\) for the set of variables in the domain of \(\sigma\) and \( \text{ran}(\sigma)\) for the set of 
expressions od \(\sigma\), ie:

\begin{alignat*}{2}
    & \text{dom}(\{e_1 / x_1, \dots, e_k / x_k\}) \ \ &&= \ \ \{x_{1}, \dots, x_{k}\}\\
    & \text{ran}(\{e_1 / x_1, \dots, e_k / x_k\}) \ \ &&= \ \ \{e_{1},\, \dots,\:\! e_{k}\}\\
\end{alignat*}

\newpage

Define the application of simultaneous substitution to a term by:
\begin{alignat*}{3}
    &\sigma x &&=\begin{cases}
        \ \sigma x & \text {if } x \in \operatorname{dom}(\sigma) \\
        \ x \qquad & \text {otherwise }
    \end{cases}\\
    &\sigma n &&= \ \ n \\
    &\sigma(b) &&= \ \ b \\
    &\sigma(\text{\textbf{skip}}) &&=\ \ \text{\textbf{skip}} \\
    &\sigma(!\ell) &&= \ \ \ !\ell \\
    &\sigma(\ell:=e) &&= \ \ \ell:=\sigma(e) \\
    &\sigma\left(e_1\ e_2\right) &&= \ \ \left(\sigma\ e_1\right)\left(\sigma\ e_2\right) \\
    &\sigma\left(e_1 ; e_2\right) &&= \ \ \sigma\left(e_1\right);\ \sigma\left(e_2\right) \\
    &\sigma\left(e_1 \text{\textbf{ op }}  e_2\right) &&= \ \ \sigma\left(e_1\right) \text{\textbf{ op }} \sigma\left(e_2\right) \\
    &\sigma\left(\text{\textbf{if }} e_1\ \text{\textbf{then}}\ e_2\ \text{\textbf{else}}\ e_3\right)\ \ &&= \ \ \text{\textbf{if}}\ \sigma\left(e_1\right)\ \text{\textbf{then}}\ \sigma\left(e_2\right)\ \text{\textbf{else}}\ \sigma\left(e_3\right) \\
    &\sigma\left(\text{\textbf{while}}\ e_1\ \text{\textbf{do}}\ e_2\right) &&= \ \ \text{\textbf{while}}\ \sigma\left(e_1\right)\ \text{\textbf{do}}\ \sigma\left(e_2\right)\\
    &\sigma(\mathbf{fn}\ x: T \Rightarrow e) &&=\ \ \mathbf{fn}\ x: T \Rightarrow(\sigma\ e) \quad &&\,\text { if } x \notin \operatorname{dom}(\sigma) \text { and } x \notin \text{\textbf{fv}}(\operatorname{ran}(\sigma))\left(^*\right)
\end{alignat*}

Where \((^*)\) is similar to the non-simultaneous case, where we would choose an alpha-variant to make it true.

\vspace{10px}

\subsection{Function Behaviour}

\subsubsection{Call-by-Value (CBV), what we use}

For this method we reduce the LHS of an application to a \(\text{\textbf{fn}}\)-term (which is now a value),
then arguments to values, then replace all occurrences of the formal parameter in the \(\text{\textbf{fn}}\) term by that value

\[
    v ::= b \ |\ n\ |\ \text{\textbf{skip}} \ |\  \text{\textbf{fn}}\ x: T \Rightarrow e
\]

\begin{alignat*}{2}
    &( \text{app}1) \quad &&\begin{prooftree}
        \hypo{\langle e_{1}, s \rangle \longrightarrow \langle e_{1}', s' \rangle}
        \infer1{\langle e_{1}\ e_{2}, s \rangle \longrightarrow \langle e_1'\ e_{2}, s' \rangle}
    \end{prooftree}\\[10px]
    &( \text{app}2) \quad &&\begin{prooftree}
        \hypo{\langle e_{2}, s \rangle \longrightarrow \langle e_{2}', s' \rangle}
        \infer1{\langle v\ e_{2}, s \rangle \longrightarrow \langle v\ e_{2}', s' \rangle}
    \end{prooftree}
\end{alignat*}

\[
    ( \text{fn}) \quad \langle (\text{\textbf{fn}}\ x: T \Rightarrow e)\ v, s \rangle \longrightarrow  \langle \{v / x\}e, s \rangle  
\]

\subsubsection{Call-by-Name (CBN)}

For this method we reduce the LHS of an application to a \(\text{\textbf{fn}}\)-term (which is now a value),
then replace all occurrences of the formal parameter in the \(\text{\textbf{fn}}\) term by the argument

\[
    v ::= b \ |\ n\ |\ \text{\textbf{skip}} \ |\  \text{\textbf{fn}}\ x: T \Rightarrow e
\]

\[
    ( \text{CBN-app}) \quad\begin{prooftree}
        \hypo{\langle e_{1}, s \rangle \longrightarrow \langle e_{1}', s' \rangle}
        \infer1{\langle e_{1}\ e_{2}, s \rangle \longrightarrow \langle e_1'\ e_{2}, s' \rangle}
    \end{prooftree}
\]

\[
    ( \text{fn}) \quad \langle (\text{\textbf{fn}}\ x: T \Rightarrow e)\ e_2, s \rangle \longrightarrow  \langle \{e_2 / x\}e, s \rangle  
\]

\newpage

\subsubsection{Full Beta}

Allow both left and right-hand sides of application to reduce. At any point where the LHS has reduced to a \(\text{\textbf{fn}}\)-term, replace all
occurences of the formal parameter in the \(\text{\textbf{fn}}\)-term by the argument.

\vspace{5px}

Allow reductions inside lambdas.

\begin{alignat*}{2}
    &( \text{beta-app}1) \quad &&\begin{prooftree}
        \hypo{\langle e_{1}, s \rangle \longrightarrow \langle e_{1}', s' \rangle}
        \infer1{\langle e_{1}\ e_{2}, s \rangle \longrightarrow \langle e_1'\ e_{2}, s' \rangle}
    \end{prooftree}\\[10px]
    &( \text{beta-app}2) \quad &&\begin{prooftree}
        \hypo{\langle e_{2}, s \rangle \longrightarrow \langle e_{2}', s' \rangle}
        \infer1{\langle e_{1}\ e_{2}, s \rangle \longrightarrow \langle e_{1}\ e_{2}', s' \rangle}
    \end{prooftree}
\end{alignat*}

\begin{alignat*}{2}
    &( \text{beta-fn}1) \quad &&\langle (\text{\textbf{fn}}\ x: T \Rightarrow e)\ e_2, s \rangle \longrightarrow  \langle \{e_2 / x\}e, s \rangle\\[10px]
    &( \text{beta-fn}2) \quad &&\begin{prooftree}
        \hypo{\langle e, s \rangle \longrightarrow \langle e', s' \rangle}
        \infer1{\langle (\text{\textbf{fn}}\ x: T \Rightarrow e), s \rangle \longrightarrow  \langle (\text{\textbf{fn}}\ x: T \Rightarrow e'), s \rangle}
    \end{prooftree}
\end{alignat*}

This reduction relation includes the CBV and CBN relations, and also reduction inside lambdas.

\vspace{5px}

We could also do \textbf{Normal-order reduction} as seen in computation theory, (leftmost, outermost variant of full beta).


\vspace{20px}

We use \textbf{Call-By-Value} from now on

\vspace{20px}

\subsection{Local Definitions and Recursive Functions}

For readability, we want to be able to name definitions, and to restrict their scope, so we add

\[
    \text{\textbf{let val }} x: T = e_{1}\ \text{\textbf{in}}\ e_{2}\ \text{\textbf{end}}
\]

This \(x\) is a binder, binding any free occurences of \(x\) in \(e_2\).

\vspace{5px}

Can regard this as syntactic sugar:

\[
    \text{\textbf{let val }} x: T = e_{1}\ \text{\textbf{in}}\ e_{2}\ \text{\textbf{end}}\ \ \rightsquigarrow\ \ \big(\text{\textbf{fn }} x: T \Rightarrow e_{2}\big) e_{1}
\]

Our alpha naming convention means that this is really a local definition - there is no way to refer to the locally-defined variable outside of the \textbf{let val}.


To allow for recursive definitions we could add something like:

\[
    \text{\textbf{let val rec }} x: T = e\ \text{\textbf{in}}\ e'\ \text{\textbf{end}}
\]

Where \(x\) binds in both \(e\) and \(e'\). But this leads to some weird things in a CBV language.

\[
    \text{\textbf{let val rec }} x: \text{int} = 3 :: x\ \text{\textbf{in}}\ x\ \text{\textbf{end}}
\]

In a CBN language, it is reasonable to allow this kind of thing, as will only compute as much as needed. In a CBV language, would usually disallow,
allowing recursive definitions only of functions. So we need to specialise the \textbf{let val rec } construct to only allow recursion at function types
and only of function values.

\[
    \text{\textbf{let val rec }} x: T_{1} \to T_{2} = \big(\text{\textbf{fn}}\ y: T_{1} \Rightarrow e_{1}\big)\ \text{\textbf{in}}\ e_{2}\ \text{\textbf{end}}
\]

Here, the \(y\) binds in \(e_1\); the \(x\) binds in \(\big(\text{\textbf{fn}}\ y: T_{1} \Rightarrow e_{1}\big)\) and \(e_2\)

\newpage

\subsection{Full L2 Operational Semantics}\label{L2-collected-semantics}

Additions to \fullref{L1-collected-semantics}

\begin{alignat*}{2}
    &( \text{app}1) \quad &&\begin{prooftree}
        \hypo{\langle e_{1}, s \rangle \longrightarrow \langle e_{1}', s' \rangle}
        \infer1{\langle e_{1}\ e_{2}, s \rangle \longrightarrow \langle e_1'\ e_{2}, s' \rangle}
    \end{prooftree}\\[10px]
    &( \text{app}2) \quad &&\begin{prooftree}
        \hypo{\langle e_{2}, s \rangle \longrightarrow \langle e_{2}', s' \rangle}
        \infer1{\langle v\ e_{2}, s \rangle \longrightarrow \langle v\ e_{2}', s' \rangle}
    \end{prooftree}
\end{alignat*}

\[
    ( \text{fn}) \quad \langle (\text{\textbf{fn}}\ x: T \Rightarrow e)\ v, s \rangle \longrightarrow  \langle \{v / x\}e, s \rangle  
\]

\[
    \begin{prooftree}
        \hypo{\langle e_{1}, s \rangle \longrightarrow \langle e_{1}', s' \rangle  }
        \infer[left label = (let1)]1{\langle \text{\textbf{let}}\ \text{\textbf{val}}\ x: T = e_{1}\ \text{\textbf{in}}\ e_{2}\ \text{\textbf{end}}, s \rangle \longrightarrow \langle \text{\textbf{let}}\ \text{\textbf{val}}\ x: T = e_{1}'\ \text{\textbf{in}}\ e_{2}\ \text{\textbf{end}}, s' \rangle}
    \end{prooftree}
\]

\vspace{5px}

\[
    ( \text{let}2) \quad \langle \text{\textbf{let}}\ \text{\textbf{val}}\ x: T = v\ \text{\textbf{in}}\ e_{2}\ \text{\textbf{end}}, s \rangle \longrightarrow \langle \{v / x\}e_{2}, s \rangle
\]
\begin{alignat*}{1}
    ( \text{letrecfn}) \quad &\langle \text{\textbf{let val rec}}\ x: T_{1} \to T_2 = (\text{\textbf{fn}}\ y: T_1 \Rightarrow e_{1})\ \text{\textbf{in}}\ e_{2}\ \text{\textbf{end}}, s \rangle\\
    \longrightarrow\ \ &\langle \{(\text{\textbf{fn}}\ y: T_1 \Rightarrow \text{\textbf{let val rec}}\ x: T_{1} \to T_2 = (\text{\textbf{fn}}\ y: T_1 \Rightarrow e_{1})\ \text{\textbf{in}}\ e_{1}\ \text{\textbf{end}}) / x\} e_{2}, s \rangle
\end{alignat*}

\newpage


\subsection{Function Typing}

Before, \(\Gamma\) gave the types of store locations; it ranged over \(\text{\textbf{TypeEnv}}\) which was the set of all finite partial functions
from locations \(\mathbb{L}\) to \(T_{loc}\).

\vspace{5px}

Now, it must also give assumptions on the types of variables:

Type environments \(\Gamma\) are now pairs of a \(\Gamma_{loc}\) and \(\Gamma_{var}\) with the latter being a partial function from \(\mathbb{X}\) to \(T\).

\vspace{10px}

We write 
\[
    \text{dom}(\Gamma) = \text{dom}(\Gamma_{loc})\ \ \cup\ \ \text{dom}(\Gamma_{var})
\]
If \(x \not\in \text{dom}(\Gamma_{var})\), write \(\Gamma, x:T\) for the pair of \(\Gamma_{loc}\) and the partial function which maps \(x\) to \(T\)
but otherwise is like \(\Gamma_{var}\).

\begin{theo}[Progress]{}
    If \(e\) closed and \(\Gamma \vdash e: T\) and \( \text{dom}(\Gamma) \subseteq \text{dom}(s)\) then either \(e\) is a value or there exists
    \(e', s'\) such that \(\langle e,s \rangle \longrightarrow \langle e', s' \rangle\)
\end{theo}

\begin{theo}[Type Preservation]{}
    If \(e\) closed and \(\Gamma \vdash e: T\) and \( \text{dom}(\Gamma) \subseteq \text{dom}(s)\) and \(\langle e,s \rangle \longrightarrow \langle e', s' \rangle\)
    then \(\Gamma \vdash e': T\) and \(e'\) closed \( \text{dom}(\Gamma) \subseteq \text{dom}(s')\)
\end{theo}


\begin{theo}[Normalisation]{}
    In the sublanguage without while loops or store operations, if \(\Gamma \vdash e: T\) and \(e\) closed then ther does not exist an infinite
    reduction sequence \(\langle e, \{\} \rangle \longrightarrow \langle e_{1}, \{\} \rangle \longrightarrow \langle e_{2}, \{\} \rangle \longrightarrow \dots\)
\end{theo}

\subsection{L2 Collected Typing System}\label{L2-collected-types}

Additions to \fullref{L1-collected-types}

\vspace{10px}

Type environments \(\Gamma\) are now pairs of a \(\Gamma_{loc}\) and \(\Gamma_{var}\)

\[
\begin{gathered}
\text { (var) } \quad \Gamma \vdash x: T \quad \text { if } \Gamma(x)=T \\[10px]
(\mathrm{fn}) \quad \frac{\Gamma, x: T \vdash e: T^{\prime}}{\Gamma \vdash \text{\textbf{fn}}\ x: T \Rightarrow e: T \rightarrow T^{\prime}} \\[10px]
(\mathrm{app})\quad \frac{\Gamma \vdash e_1: T \rightarrow T^{\prime} \quad \Gamma \vdash e_2: T}{\Gamma \vdash e_1 e_2: T^{\prime}} \\[10px]
\text { (let) }\quad \frac{\Gamma \vdash e_1: T \quad \Gamma, x: T \vdash e_2: T^{\prime}}{\Gamma \vdash \text{\textbf{let val }} x: T=e_1 \text{ \textbf{in} } e_2\ \text{\textbf{end}}:T^{\prime}} \\[10px]
\text { (let rec fn) }\quad \frac{\Gamma, x: T_1 \rightarrow T_2, y: T_1 \vdash e_1: T_2 \quad \quad \quad \Gamma, x: T_1 \rightarrow T_2 \vdash e_2: T}{\Gamma \vdash \text {\textbf{  let val rec }} x: T_1 \rightarrow T_2=\left(\text{\textbf{fn}}\ y: T_1 \Rightarrow e_1\right)\ \text{\textbf{in}}\ e_2\ \text{\textbf{end}}: T}
\end{gathered}
\]

\newpage

\section{Data - L3}

\subsection{Products and Sums}

The \textbf{Product} type \(\ \ T_{1} * T_2\ \ \) lets you tuple together values of types \(T_1\) and \(T_2\).

\vspace{5px}

A product may be constructed by simply using parentheses around two expressions seperated by a comma.
\[
    (e_{1}, e_{2})
\]

Individual elements may be projected out of the product using either \(\#1\) or \(\#2\) for projecting the first or second expression out of the 
product respectively. This may be seen as destructing the product type.
\[
    \langle \#1 (v_1, v_2), s \rangle \longrightarrow \langle v_{1}, s \rangle
\]

We however do not allow for \(\#e\ e'\) as this cannot be typechecked.

\vspace{5px}

Before we allow for data to be projected out of the product we must reduce the product down to a product of two values, (which itself is now a value).
We do this with left to right reductions like the rest of the language.

\[
    \begin{prooftree}
        \hypo{\langle e, s \rangle \longrightarrow \langle e', s' \rangle}
        \infer1{\langle \#1\ e, s \rangle \longrightarrow \langle \#1\ e', s \rangle}
    \end{prooftree}
\]

This will reduce down until the product is a value, meaning it only contains other values. (Similar for \#2)

\vspace{30px}

The \textbf{Sum} type \(\ \ T_1 + T_2\ \ \) lets us form a disjoint union, with a value of the sum type either being being a value of type \(T_1\) or a value of type \(T_2\).

\vspace{5px}

We construct a product type by injecting a value into the type, either inject left or inject right to create a value of type \(T_1\) or \(T_2\) respectively.
\[
    \text{\textbf{inl}}\ e: T \qquad \qquad \qquad \text{\textbf{inr}}\ e: T
\]

We specify the type that the expression is being injected into to maintain the unique typing property. This specifiying the type is not the type 
inference but part of the semantics. If we did not specify then then we would have:
\[
    \{\} \vdash \text{\textbf{inl}}\ 3: \text{int} + \text{int} \quad \land \quad \{\} \vdash \text{\textbf{inl}}\ 3: \text{int} + \text{bool} \quad \land \quad \dots
\]

A compiler might use a type inference algorithm that can infer the type that it should be.

\vspace{5px}

We can determine what value a sum type is by case splitting, this can also be seen as destructing the sum type.
\[
    \text{\textbf{case}}\ e :T\ \text{\textbf{of  inl}}\ (x: T_1) \Rightarrow e_{1} \ |\ \text{\textbf{inr}}\ (y:T_{2}) \Rightarrow e_{2}
\]


\vspace{10px}

\begin{center}
    \begin{tabular}{| l | c | c |}
        \hline
        Type & Constructors & Destructors\\[2px]
        \hline
        & &\\[-5px]
        \(T \to T\) & \(\text{\textbf{fn}}\ x:T \Rightarrow \dunder\) & \(\dunder\ e\)\\[5px]
        \(T * T\) & \((\dunder, \dunder)\) & \(\#1\ \dunder \quad \lor \quad \#2\ \dunder\)\\[5px]
        \(T + T\) & \(\text{\textbf{inl}}\ (\dunder) \quad \lor \quad \text{\textbf{inr}}\ (\dunder)\) & \(\text{\textbf{case}}\)\\[5px]
        \( \text{bool}\) & \(\text{\textbf{true}} \quad \lor  \quad \text{\textbf{false}}\) & \(\text{\textbf{if}}\)\\
        \hline
    \end{tabular}
\end{center}

\newpage

\subsection{Datatypes and Records}

\textbf{Records} are a generalisation of products. Labels \(lab \in \mathbb{LAB}\) for a set \(\mathbb{LAB} = \{p,q, \dots\}\). Each label
has a corresponding expression associated with it, which can be projected out using the label.

\[
    \#q\ \{p:(x+2), q:(y+1), r:(5)\} \longrightarrow^* y+1
\]

\subsection{Mutable Store}

We are changing how the store works now from L1 and L2, in those languages we could only store integers, but we would like to store any value. We
also cannot create any new locations during the runtime and thus they must all exist at the beginning. Functions cannot also abstact on
locations.

\vspace{5px}

We remove the specific intref type and replace the type with a \(T\) ref type, and remove the specific assign and deref rules in favour of
assigning an expression to another, and derefencing an expression etc., making everything more general.

\vspace{5px}

We make locations variables now, and the store \(s\) was a finite partial map from \(\mathbb{L}\) to \(\mathbb{Z}\) now we take stores to be
the finite partial map from \(\mathbb{L}\) to the set of all values.

\vspace{10px}

\subsection{Full L3 Operational Semantics}

Additions to \fullref{L2-collected-semantics}

\begin{alignat*}{2}
    &\begin{prooftree}
        \hypo{\langle e_{1}, s \rangle \longrightarrow \langle e_{1}', s' \rangle}
        \infer[left label = (pair1)]1{\langle (e_{1}, e_{2}), s \rangle \longrightarrow \langle (e_{1}', e_{2}), s' \rangle }
    \end{prooftree}\\[10px]
    &\begin{prooftree}
        \hypo{\langle e_{2}, s \rangle \longrightarrow \langle e_{2}', s' \rangle}
        \infer[left label = (pair2)]1{\langle (v, e_{2}), s \rangle \longrightarrow \langle (v, e_{2}'), s' \rangle }
    \end{prooftree}\\[10px]
    &( \text{proj}1)\ \, \langle \#1(v_{1}, v_{2}), s \rangle \longrightarrow \langle v_{1}, s' \rangle \qquad &&( \text{proj}2)\ \, \langle \#2(v_{1}, v_{2}), s \rangle \longrightarrow \langle v_{2} s' \rangle \\[10px]
    &\begin{prooftree}
        \hypo{\langle e, s \rangle \longrightarrow \langle e', s' \rangle}
        \infer[left label = (proj3)]1{\langle \#1\ \ e, s \rangle \longrightarrow \langle \#1\ \ e', s' \rangle }
    \end{prooftree} &&\begin{prooftree}
        \hypo{\langle e, s \rangle \longrightarrow \langle e', s' \rangle}
        \infer[left label = (proj4)]1{\langle \#2\ \ e, s \rangle \longrightarrow \langle \#2\ \ e', s' \rangle }
    \end{prooftree}\\[10px]
    &\ \ \ \, \begin{prooftree}
        \hypo{\langle e, s \rangle \longrightarrow \langle e', s' \rangle}
        \infer[left label = (inl)]1{\langle \text{\textbf{inl}}\ e:T, s \rangle \longrightarrow \langle  \text{\textbf{inl}}\ e':T, s' \rangle }
    \end{prooftree}
    &&\ \ \ \, \begin{prooftree}
        \hypo{\langle e, s \rangle \longrightarrow \langle e', s' \rangle}
        \infer[left label = (inl)]1{\langle \text{\textbf{inl}}\ e:T, s \rangle \longrightarrow \langle  \text{\textbf{inl}}\ e':T, s' \rangle }
    \end{prooftree}\\
\end{alignat*}
\[
    \begin{prooftree}
        \hypo{\langle e, s \rangle \longrightarrow \langle e', s' \rangle}
        \infer[left label = (case1)]1{\begin{matrix}\ \ \ \ \, \langle \text{\textbf{case}}\ e\phantom{'} :T\ \text{\textbf{of  inl}}\ (x: T_1)  \Rightarrow e_{1} \ |\ \text{\textbf{inr}}\ (y:T_{2}) \Rightarrow e_{2}, s \rangle\\ \longrightarrow \langle \text{\textbf{case}}\ e' :T\ \text{\textbf{of  inl}}\ (x: T_1)  \Rightarrow e_{1} \ |\ \text{\textbf{inr}}\ (y:T_{2}) \Rightarrow e_{2}, s' \rangle \end{matrix}}
    \end{prooftree}
\]

\[
    \text{(case2)}\ \ \langle \text{\textbf{case}}\ \text{\textbf{inl}}\ v :T\ \text{\textbf{of  inl}}\ (x: T_1)  \Rightarrow e_{1} \ |\ \text{\textbf{inr}}\ (y:T_{2}) \Rightarrow e_{2}, s \rangle \longrightarrow \langle \{v / x\}e_{1}, s \rangle 
\]
\[
    \text{(case3)}\ \ \langle \text{\textbf{case}}\ \text{\textbf{inr}}\ v :T\ \text{\textbf{of  inl}}\ (x: T_1)  \Rightarrow e_{1} \ |\ \text{\textbf{inr}}\ (y:T_{2}) \Rightarrow e_{2}, s \rangle \longrightarrow \langle \{v / y\}e_{2}, s \rangle 
\]

\newpage

\subsection{Type Checking The Store}

For L1, our type properties used \( \text{dom}(\Gamma) \subseteq \text{dom}(s)\) to express the condition `All locations mentioned in \(\Gamma\) exist in the 
store s'.

\vspace{5px}

Now we require more:

\begin{itemize}
    \item For each \(\ell \in \text{dom}(s)\) need that \(s(\ell)\) is typable 
    \item \(s(\ell )\) might contain some other locations
\end{itemize}

\begin{defin}[Well-typed Store]{}
    Let \(\Gamma \vdash s\) if \( \text{dom}(\Gamma) = \text{dom}(s)\) and if for all \(\ell  \in \text{dom}(s)\), if \(\Gamma(\ell ) = T\ \text{ref}\)
    then \(\Gamma \vdash s(\ell ): T\).
\end{defin}

\begin{theo}[Progress]{}
    If \(e\) closed and \(\Gamma \vdash e: T\) and \(\Gamma \vdash e: T\) and \(\Gamma \vdash s\) then either \(e\) is a value or there exists \(e', s'\)
    such that \(\langle e,s \rangle \longrightarrow \langle e', s' \rangle\).
\end{theo}

\begin{theo}[Type Preservation]{}
    If \(e\) closed and \(\Gamma \vdash e: T\) and \(\Gamma \vdash s\) and \(\langle e,s \rangle \longrightarrow \langle e', s' \rangle\) then \(e'\) is 
    closed and for some \(\Gamma'\) with disjoint domain to \(\Gamma\) we have \(\Gamma, \Gamma' \vdash e': T\) and \(\Gamma, \Gamma' \vdash s\).
\end{theo}

\begin{theo}[Type Safety]{}
    If \(e\) closed and \(\Gamma \vdash e: T\) and \(\Gamma \vdash s\) and \(\langle e,s \rangle \longrightarrow^* \langle e', s' \rangle\) then either \(e'\) is 
    a value or there exists \(e'', s''\), such that \(\langle e',s' \rangle \longrightarrow  \langle e'', s'' \rangle \)
\end{theo}

\newpage

\subsection{Evaluation Contexts}

Instead of having multiple individual rules we can create an evaluation context. This is especially beneficial for larger languages
but not necessary for the L languages described here. Define evaluation contexts:
\begin{alignat*}{1}
    E\ \ ::=\ \ &\dunder\ \ op\ \ e \ |\ v\ \ op\ \ \dunder \ |\ \text{\textbf{if}}\ \dunder\ \text{\textbf{then}}\ e\ \text{\textbf{else}}\ e \ |\ \\
    &\dunder;\ e \ |\ \\
    &\dunder\ \ e \ |\ v \ \ \dunder \\
    & \text{\textbf{let}}\ \text{\textbf{val}}\ x:T = \dunder\ \text{\textbf{in}}\ e_{2}\ \text{\textbf{end}} \ |\ \\
    & (\dunder,\ e) \ |\ (v, \dunder) \ |\ \#1\ \dunder \ |\  \#2\ \dunder \ |\ \\
    & \text{\textbf{inl}}\ \dunder\ \text{\textbf{of}}\ \text{\textbf{inl}}\ (x:T) \Rightarrow e \ | \ \text{\textbf{inr}}\ (x:T) \Rightarrow e \ |\ \\
    & \{lab_1 = v_{1}, \dots,\ lab_i = \dunder,\ \dots,\ lab_k = e_k\}\ |\ \#lab\ \ \dunder\ |\ \\
    & \dunder \coloneq e \ |\ v \coloneq \dunder \ |\ !\dunder \ |\ \text{ref}\ \dunder
\end{alignat*}

And we now have the single context rule:
\[
    \begin{prooftree}
        \hypo{\langle e,s \rangle \longrightarrow \langle e', s' \rangle}
        \infer[left label = (eval)\(\ \)]1{\langle E[e], s \rangle \longrightarrow \langle E[e'], s' \rangle}
    \end{prooftree}
\]
\[
    (\text{Where } E[e] \text{ means that we substitute an expression } e \text{ into on the holes } \dunder \text{ in the above evaluation context})
\]

We replace all the rules with premises with the above rule, we also add the computation rules as well 

( (op+), (op\( \geq\)), (if1), (if2), ... )


\newpage

\section{Subtyping and Objects}

The type system so far is very rigid we would like to introduce Subtype Polymorphism.

\vspace{10px}

Any value of type \(\{p: \text{int}, q: \text{int}\}\) can be used wherever a value of type \(\{p: \text{int}\}\) is expected.

\vspace{5px}

We introduce a subtyping relation between types, written \(T <: T'\), read as \(T\) is a subtype of \(T'\)

\[
    \{p: \text{int}, q: \text{int}\} \ <: \ \{p: \text{int}\} \ <: \ \{\}
\]

Introduce a subsumption rule

\[
    \begin{prooftree}
        \hypo{\Gamma \vdash e: T}
        \hypo{T <: T'}
        \infer[left label = (sub)]2{\Gamma \vdash e: T'}
    \end{prooftree}
\]

The Subtype Relation

\[
    \begin{prooftree}
        \hypo{}
        \infer[left label = (s-refl)]1{T <: T}
    \end{prooftree}
\]

\[
    \begin{prooftree}
        \hypo{T <: T'}
        \hypo{T' <: T''}
        \infer[left label = (s-trans)]2{T <: T''}
    \end{prooftree}
\]

\[
    ( \text{s-record-width}) \ \ \{lab_1:T_1, \dots, lab_n:T_n, \dots, lab_{n+k}: T_{n+k}\} <: \{lab_1:T_1, \dots, lab_n:T_n\}
\]

\[
    \begin{prooftree}
        \hypo{T_1 <: T_1'}
        \hypo{\dots}
        \hypo{T_n <: T_n'}
        \infer[left label = (s-record-depth)]3{\{lab_1:T_1, \dots, lab_n:T_n\} <: \{lab_1:T_1', \dots, lab_n:T_n'\}}
    \end{prooftree}
\]

\[
    \begin{prooftree}
        \hypo{\pi \text{ a permutation of } 1, \dots, n}
        \infer[left label = (s-record-depth)]1{\{lab_1:T_1, \dots, lab_n:T_n\} <: \{lab_{\pi(1)}:T_{\pi(1)}, \dots, lab_{\pi(n)}:T_{\pi(n)}\}}
    \end{prooftree}
\]

\[
    \begin{prooftree}
        \hypo{T_1' <: T_{1}}
        \hypo{T_2 <: T_{2}'}
        \infer[left label = (s-fn)]2{T_1 \to T_2 <: T_{1}' \to T_{2}'}
    \end{prooftree}
\]

\[
    \begin{prooftree}
        \hypo{T_1 <: T_{1}'}
        \hypo{T_2 <: T_{2}'}
        \infer[left label = (s-pair)]2{T_1 * T_2 <: T_{1}' * T_{2}'}
    \end{prooftree}
\]

\[
    \begin{prooftree}
        \hypo{T_1 <: T_{1}'}
        \hypo{T_2 <: T_{2}'}
        \infer[left label = (s-pair)]2{T_1 * T_2 <: T_{1}' * T_{2}'}
    \end{prooftree}
\]

\newpage

\section{Concurrency}

\begin{alignat*}{1}
    e\ \ ::=\ \ &n \ |\  b \ |\  e_{1} \ op \ e_{2} \ |\  \text{\textbf{if}}\ e_{1} \text{\textbf{then}}\ e_{2}\ \text{\textbf{else}}\ e_{3} \ |\ \\
    &\ell \coloneq e \ |\  !l \ |\ \\
    &\text{\textbf{()}} \ |\  e_{1};e_{2} \ |\ \\
    &\text{\textbf{while}}\ e_{1}\ \text{\textbf{do}}\ e_{2} \ |\ \\
    & e_{1} | e_{2}
\end{alignat*}

\begin{alignat*}{2}
    &T &&::=\ \ \text{int} \ |\ \text{bool} \ |\ \text{unit} \ |\ \text{proc}\phantom{aaaaaaaaaaaaaaaa}\\
    &T_{loc}\ \ &&::=\ \ \text{intref}
\end{alignat*}

\begin{alignat*}{1}
    &\begin{prooftree}
        \hypo{\Gamma \vdash e: \text{unit}}
        \infer[left label = (thread)]1{\Gamma \vdash e: \text{proc}}
    \end{prooftree}\\[10px]
    &\begin{prooftree}
        \hypo{\Gamma \vdash e_{1}: \text{proc}}
        \hypo{\Gamma \vdash e_{2}: \text{proc}}
        \infer[left label = (parallel)]2{\Gamma \vdash e_{1}|e_{2}: \text{proc}}
    \end{prooftree}
\end{alignat*}

\begin{alignat*}{1}
    &\begin{prooftree}
        \hypo{\langle e_1, s \rangle \longrightarrow \langle e_{2}, s \rangle}
        \infer[left label = (parallel1)]1{\hypo{\langle e_1 | e_{2}, s \rangle \longrightarrow \langle e_{1}' | e_{2}, s \rangle}}
    \end{prooftree}\qquad\ \\[10px]
    &\begin{prooftree}
        \hypo{\langle e_2, s \rangle \longrightarrow \langle e_{2}', s \rangle}
        \infer[left label = (parallel1)]1{\hypo{\langle e_1 | e_{2}, s \rangle \longrightarrow \langle e_{1} | e_{2}', s \rangle}}
    \end{prooftree}
\end{alignat*}

\subsection{Mutexes}

Mutex names \(m \in \mathbb{M} = \{m, m_{1}, \dots, m_{n}\}\)

\vspace{5px}

Configurations \(\langle e, s, M \rangle\) where \(M: \mathbb{M} \longrightarrow \mathbb{B}\) is the mutex state

\vspace{5px}

Expressions \(e ::= \dots \ |\ \text{\textbf{lock}}\ m \ |\ \text{\textbf{unlock}}\ m\)

\[
    \begin{prooftree}
        \hypo{}
        \infer[left label = (lock)]1{\Gamma \vdash \text{\textbf{lock}}\ m : \text{unit}}
    \end{prooftree}\qquad\qquad\qquad
    \begin{prooftree}
        \hypo{}
        \infer[left label = (unlock)]1{\Gamma \vdash \text{\textbf{unlock}}\ m : \text{unit}}
    \end{prooftree}
\]

\[
    ( \text{lock}) \ \ \langle \text{\textbf{lock}}\ m, s, M \rangle  \longrightarrow  \langle (), s, M + \{m \mapsto \text{\textbf{true}}\} \rangle \quad \text{if}\ \neg M(m)
\]

\[
    ( \text{unlock}) \ \ \langle \text{\textbf{unlock}}\ m, s, M \rangle  \longrightarrow  \langle (), s, M + \{m \mapsto \text{\textbf{false}}\} \rangle
\]

lock atomically checks the mutex is currently false and changes its state and lets the thread proceed.

\vspace{10px}

All other rules need to carry the mutex state around and the changes from premises to its transition.


\subsection{An Ordered 2PL Discipline, Informally}

Fix an association between locations and mutexes. For simplicity make it 1:1 - \(\ell\) with \(m\), \(l_1\) with \(m_1\) etc.

\vspace{5px}

Fix a lock acquisition order. For simplicity, make it \(m, m_0, m_1, m_2, \dots\)

\vspace{5px}

Require that each \(e_i\)

\begin{itemize}
    \item acquires the lock \(m_j\) for each location \(l_j\) it uses, before it uses it 
    \item acquires and releases each lock in a properly bracketed way 
    \item does not acquire any lock after it's released any lock 
    \item acquires locks in increasing order
\end{itemize}

Then any concurrent state should never deadlock and be serializable - any execution of it should be `equivalent' to an execution of some permutaion
that is a sequence of the expressions sequentially.

\newpage

\section{Semantic Equivalence}

Take \(\simeq\) to the semantic equivalence relation symbol. For \(\simeq\) to be good it must obey

\vspace{5px}

1. Programs that result in observably different values from some initial store are not equivalent.
\[
    \big( \exists s, s_{1}, s_{2}, v_{1}, v_{2}. \langle e_{1},s \rangle \longrightarrow^* \langle v_{1}, s_{1} \rangle \land \langle e_{2}, s \rangle \longrightarrow^* \langle v_{2}, s_{2} \rangle \land v_{1} \neq v_{2}\big) \implies e_{1} \not\simeq e_{2}
\]

2. Programs that terminate must not be equivalent to programs that do not

\vspace{5px}

3. \(\simeq\) must be an equivalence relation
\[
    e \simeq e \qquad e_1 \simeq e_{2} \implies e_{2} \simeq e_{1} \qquad e_{1} \simeq e_{2} \simeq e_{3} \implies e_{1} \simeq e_{3}
\]

4. \(\simeq\) must be a congruence
\[
    \text{if}\ e_{1} \simeq e_{2} \text{ then for any context } C \text{ we must have } C[e_{1}] \simeq C[e_{2}]
\]

\vspace{5px}

5. \(\simeq\) should relate as many programs as possible subject to the above

\vspace{10px}

\subsection{Semantic Equivalence for L1}

Consider Typed L1 again.

\vspace{5px}

Define \(e_{1} \simeq_\Gamma^T e_{2}\) to hold iff forall \(s\) such that \( \text{dom}(\Gamma) \subseteq \text{dom}(s)\), we have \(\Gamma \vdash e_{1}: T\), \(\Gamma \vdash e_{2}:T\) and either:

\[
    \langle e_{1}, s \rangle \longrightarrow^\omega\ \ \land\ \ \langle e_{2}, s \rangle \longrightarrow^\omega
\]
\[
    \text{or}
\]
\[
    \exists v, s'. \ \langle e_{1}, s \rangle \longrightarrow^* \langle v, s' \rangle\ \ \land\ \ \langle e_{2}, s \rangle \longrightarrow^* \langle v,s' \rangle   
\]

With the L1 contexts being:
\begin{alignat*}{1}
    C\ \ ::=\ \ &\dunder\ \ op\ \ e \ |\ e_{1}\ \ op\ \ \dunder \ |\ \\
    &\text{\textbf{if}}\ \dunder\ \text{\textbf{then}}\ e_{2}\ \text{\textbf{else}}\ e_{3} \ |\ \text{\textbf{if}}\ e_{1}\ \text{\textbf{then}}\ \dunder\ \text{\textbf{else}}\ e_{3} \ |\ \\
    &\text{\textbf{if}}\ e_{1}\ \text{\textbf{then}}\ e_{2}\ \text{\textbf{else}}\ \dunder \ |\  \\
    &\ell \coloneq \dunder \ |\ \\
    &\dunder;\ e_{2} \ |\ e_{1};\ \dunder\\
    &\text{\textbf{while}}\ \dunder\ \text{\textbf{do}}\ e_{2} \ |\ \text{\textbf{while}}\ e_{1}\ \text{\textbf{do}}\ \dunder
\end{alignat*}

Say \(\simeq_\Gamma^T\) has the congurence property if whenever \(e_{1} \simeq_\Gamma^T e_{2}\) we have, for all \(C\) and \(T'\), if \(\Gamma \vdash C[e_{1}]: T'\)
and \(\Gamma \vdash C[e_{2}]: T'\) then \(C[e_{1}] \simeq_\Gamma^{T'} C[e_{2}]\)

\vspace{10px}

\subsection{Contextual Equivalence for L3}

Consider typed L3 programs, \(\Gamma \vdash e_{1}: T\) and \(\Gamma \vdash e_{2}:T\).

\vspace{5px}

We say they are contextually equivalent if, for every context \(C\) such that \(\{\} \vdash C[e_{1}]: \text{unit}\) and \(\{\} \vdash C[e_{2}]: \text{unit}\), we either have:

\[
    \langle C[e_{1}] \{\} \rangle \longrightarrow^\omega\ \ \land\ \ \langle C[e_{2}], \{\}\rangle \longrightarrow^\omega   
\]
\[
    \text{or}
\]
\[
    \exists s_{1}, s_{2}. \langle C[e_{1}], \{\} \rangle \longrightarrow^* \langle \text{\textbf{skip}}, s_{1} \rangle\ \ \land\ \ \langle C[e_{2}], \{\} \rangle \longrightarrow^* \langle \text{\textbf{skip}}, s_{2} \rangle    
\]

\end{document}