\usepackage{hyperref}
\hypersetup{
    colorlinks,
    citecolor=black,
    filecolor=black,
    linkcolor=black,
    urlcolor=black
}

\usepackage{graphicx}
\usepackage{tikz}
\usepackage[framemethod=TikZ]{mdframed}
\usepackage{geometry}
\usepackage{amsmath}
\usepackage{amssymb}
\usepackage{amsfonts}
\usepackage{scrextend}
\usepackage{enumitem, multicol}
\usepackage{caption}
\usepackage{ifthen}
\usepackage{changepage}
\usepackage{listings}
\usepackage{tikz-cd}
\usetikzlibrary{arrows.meta, automata, 
                positioning, 
                quotes}
\usepackage{amsmath}
\usepackage[low-sup]{subdepth}
\usepackage{ebproof}
\usepackage{mdframed}
\usepackage{calc}
\usepackage{mathtools}
\usepackage{forest}
\usepackage[dvipsnames]{xcolor}


\newlength\twound
\settowidth{\twound}{\_}
\newcommand{\dunder}{\rule{2\twound}{0.4pt}}

\newcommand*{\fullref}[1]{\hyperref[{#1}]{\ref*{#1} \nameref*{#1}}}

\DeclarePairedDelimiter{\ceil}{\lceil}{\rceil}

\DeclarePairedDelimiter{\encode}{\ulcorner}{\urcorner}


\makeatletter
\newsavebox{\@brx}
\newcommand{\llangle}[1][]{\savebox{\@brx}{\(\m@th{#1\langle}\)}%
  \mathopen{\copy\@brx\kern-0.5\wd\@brx\usebox{\@brx}}}
\newcommand{\rrangle}[1][]{\savebox{\@brx}{\(\m@th{#1\rangle}\)}%
  \mathclose{\copy\@brx\kern-0.5\wd\@brx\usebox{\@brx}}}
\makeatother

\lstset{
    basicstyle=\ttfamily,
    mathescape=true,          % Enable math mode inside listings
    columns=fullflexible      % Ensure spacing is preserved
}

\newmdenv[
  topline=false,
  bottomline=false,
  rightline=false,
  skipabove=\topsep,
  skipbelow=\topsep
]{siderules}

\def\axiom{\infer0}

\usetikzlibrary{calc}
% `pathmorphing` is necessary to draw squiggly arrows.
\usetikzlibrary{decorations.pathmorphing,shapes.misc}


% A TikZ style for curved arrows of a fixed height, due to AndréC.
\tikzset{curve/.style={settings={#1},to path={(\tikztostart)
    .. controls ($(\tikztostart)!\pv{pos}!(\tikztotarget)!\pv{height}!270:(\tikztotarget)$)
    and ($(\tikztostart)!1-\pv{pos}!(\tikztotarget)!\pv{height}!270:(\tikztotarget)$)
    .. (\tikztotarget)\tikztonodes}},
    settings/.code={\tikzset{quiver/.cd,#1}
        \def\pv##1{\pgfkeysvalueof{/tikz/quiver/##1}}},
    quiver/.cd,pos/.initial=0.35,height/.initial=0}

% TikZ arrowhead/tail styles.
\tikzset{tail reversed/.code={\pgfsetarrowsstart{tikzcd to}}}
\tikzset{2tail/.code={\pgfsetarrowsstart{Implies[reversed]}}}
\tikzset{2tail reversed/.code={\pgfsetarrowsstart{Implies}}}
% TikZ arrow styles.
\tikzset{no body/.style={/tikz/dash pattern=on 0 off 1mm}}

\definecolor{dkgreen}{rgb}{0,0.6,0}
\definecolor{darkred}{rgb}{0.55,0,0}
\definecolor{gray}{rgb}{0.5,0.5,0.5}
\definecolor{mauve}{rgb}{0.58,0,0.82}

%\usetikzlibrary{decorations.markings}

\geometry{
 a4paper,
 total={170mm,257mm},
 left=30mm,
 top=25mm,
}

\newcommand{\indrule}[3]{\ifstrempty{#1}{\mathrel{\raisebox{6pt}{\(\dfrac{#2}{#3}\)}}}{\mathrel{\raisebox{6pt}{(\text{#1})\ \ \(\dfrac{#2}{#3}\)}}}}
\newcommand{\eqdef}{\overset{\text{def}}{=}}

\definecolor{darkblue}{rgb}{0,0,0.55}

\newcounter{quest}\setcounter{quest}{0}
\renewcommand{\thequest}{\arabic{quest}}
\newenvironment{quest}[1][]{%
\ifstrempty{#1}%
{%
\refstepcounter{quest}%
\mdfsetup{%
frametitle={%
\tikz[baseline=(current bounding box.east),outer sep=0pt]
\node[anchor=east,rectangle,fill=darkblue!20]
{\strut Question~\thequest.};}}
%
}%
{
\ifthenelse{\equal{#1}{a} \OR \equal{#1}{i} \OR \equal{#1}{1}}{\refstepcounter{quest}}{}
\mdfsetup{%
frametitle={%
\tikz[baseline=(current bounding box.east),outer sep=0pt]
\node[anchor=east,rectangle,fill=darkblue!20]
{\strut Question~\thequest.#1)};}}%
}%
\mdfsetup{innertopmargin=10pt,linecolor=darkblue!20,%
linewidth=2pt,topline=true,%
frametitleaboveskip=\dimexpr-\ht\strutbox\relax
}
\begin{mdframed}[]\relax%
}{\end{mdframed}}



\setlength{\parindent}{0px}

\usetikzlibrary{decorations.markings}


\geometry{
 a4paper,
 total={170mm,257mm},
 left=30mm,
 top=25mm,
}


\newcommand{\R}{\mathbb{R}}
\newcommand{\pdv}[2]{\frac{\partial #1}{\partial #2}}
\newcommand{\lbold}[1]{\textbf{\large{ #1}}}
\newcommand{\minus}{\scalebox{1.0}[1.0]{\(-\)}}
\newcommand{\boldminus}{\(\boldsymbol{-}\)}
\newcommand{\addtopstrut}{\rule[5pt]{0pt}{.7\baselineskip}}%
\newcommand{\addbottomstrut}{\rule[-5pt]{0pt}{0pt}}%

\newcounter{theo}[section]\setcounter{theo}{0}
\renewcommand{\thetheo}{\arabic{section}.\arabic{theo}}
\newenvironment{theo}[2][]{%
\refstepcounter{theo}%
\ifstrempty{#1}%
{\mdfsetup{%
frametitle={%
\tikz[baseline=(current bounding box.east),outer sep=0pt]
\node[anchor=east,rectangle,fill=red!20]
{\strut Theorem~\thetheo};}}
}%
{\mdfsetup{%
frametitle={%
\tikz[baseline=(current bounding box.east),outer sep=0pt]
\node[anchor=east,rectangle,fill=red!20]
{\strut Theorem~\thetheo:~#1};}}%
}%
\mdfsetup{innertopmargin=10pt,linecolor=red!20,%
linewidth=2pt,topline=true,%
frametitleaboveskip=\dimexpr-\ht\strutbox\relax
}
\begin{mdframed}[]\relax%
\label{#2}}{\end{mdframed}}

\definecolor{pastelgreen}{RGB}{193, 225, 193}

\newcounter{lem}[section]\setcounter{lem}{0}
\renewcommand{\thelem}{\arabic{section}.\arabic{lem}}
\newenvironment{lem}[2][]{%
\refstepcounter{lem}%
\ifstrempty{#1}%
{\mdfsetup{%
frametitle={%
\tikz[baseline=(current bounding box.east),outer sep=0pt]
\node[anchor=east,rectangle,fill=pastelgreen]
{\strut Lemma~\thelem};}}
}%
{\mdfsetup{%
frametitle={%
\tikz[baseline=(current bounding box.east),outer sep=0pt]
\node[anchor=east,rectangle,fill=pastelgreen]
{\strut Lemma~\thelem:~#1};}}%
}%
\mdfsetup{innertopmargin=10pt,linecolor=pastelgreen,%
linewidth=2pt,topline=true,%
frametitleaboveskip=\dimexpr-\ht\strutbox\relax
}
\begin{mdframed}[]\relax%
\label{#2}}{\end{mdframed}}

\definecolor{darkblue}{rgb}{0,0,0.55}

\newcounter{defin}[section]\setcounter{defin}{0}
\renewcommand{\thedefin}{\arabic{section}.\arabic{defin}}
\newenvironment{defin}[2][]{%
\refstepcounter{defin}%
\ifstrempty{#1}%
{\mdfsetup{%
frametitle={%
\tikz[baseline=(current bounding box.east),outer sep=0pt]
\node[anchor=east,rectangle,fill=darkblue!20]
{\strut Definition~\thedefin};}}
}%
{\mdfsetup{%
frametitle={%
\tikz[baseline=(current bounding box.east),outer sep=0pt]
\node[anchor=east,rectangle,fill=darkblue!20]
{\strut Definition~\thedefin:~#1};}}%
}%
\mdfsetup{innertopmargin=10pt,linecolor=darkblue!20,%
linewidth=2pt,topline=true,%
frametitleaboveskip=\dimexpr-\ht\strutbox\relax
}
\begin{mdframed}[]\relax%
\label{#2}}{\end{mdframed}}

\newcounter{prf}[section]\setcounter{prf}{0}
\renewcommand{\theprf}{\arabic{section}.\arabic{prf}}
\newenvironment{prf}[2][]{%
\refstepcounter{prf}%
\ifstrempty{#1}%
{\mdfsetup{%
frametitle={%
\tikz[baseline=(current bounding box.east),outer sep=0pt]
\node[anchor=east,rectangle,fill=red!20]
{\strut Proof~\theprf};}}
}%
{\mdfsetup{%
frametitle={%
\tikz[baseline=(current bounding box.east),outer sep=0pt]
\node[anchor=east,rectangle,fill=red!20]
{\strut Proof~\thetheo:~#1};}}%
}%
\mdfsetup{innertopmargin=10pt,linecolor=red!20,%
linewidth=2pt,topline=true,%
frametitleaboveskip=\dimexpr-\ht\strutbox\relax
}
\begin{mdframed}[]\relax%
\label{#2}}{\end{mdframed}}



\setlength{\parindent}{0px}

\tikzset{
  ctrlpoint/.style={%
    draw=gray,
    circle,
    inner sep=0,
    minimum width=1ex,
  }
}

% cSpell:disable

\newcommand\Bezier[4]{% \bezier (lowercase 'b') was already defined elsewhere
  \node (p1) [ctrlpoint,label=90:\(P_1\)] at (#1) {};
  \node (p2) [ctrlpoint,label=90:\(P_2\)] at (#2) {};
  \node (p3) [ctrlpoint,label=90:\(P_3\)] at (#3) {};
  \node (p4) [ctrlpoint,label=90:\(P_4\)] at (#4) {};
  \draw [gray] (p1) -- (p2) -- (p3) -- (p4);
  \draw [blue] (#1) .. controls (#2) and (#3) .. (#4);
}

% cSpell:enable